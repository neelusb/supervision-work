\documentclass[12pt]{article}
\usepackage[utf8]{inputenc}
\usepackage[margin=1in]{geometry}
\usepackage{amsmath}

\author{Neelu Saraswatibhatla (srns2)}
\title{MLRD Supervision 4}
\date{\vspace{-5ex}}

\begin{document}

\maketitle

\section*{Graph algorithms}

Choose a vertex on the graph at random, and run a breadth-first search with this vertex as the start vertex, keeping track of the vertex furthest away from this one. If there are any vertices in the graph that haven't been visited then there are one or more disconnected vertices, so the diameter is infinity. If all vertices are in the graph, then find the first one that is furthest away from the start vertex. Then run another breadth-first search starting from this vertex, and the distance to the furthest away vertex is the diameter of the graph.

This is because in the first BFS, if a vertex that is the start or end point of longest shortest path on the graph was picked, then one on the other end will be found to be the furthest away, and the second BFS will find its way back to the first one (or one equally distant), finding the longest shortest path. If the vertex isn't on the longest shortest path, then the vertex found by the first BFS will be an endpoint of a longest shortest path, and the second BFS will once again find the distance of this path.

This does not work on graphs with cycles. For example, consider a diamond-shaped graph with 4 vertices (1, 2, 3, 4) and edges (1, 2), (1, 3), (2, 3), (2, 4), and (3, 4). Let the first BFS start at 2. All other the vertices are at a distance of 1, so let 3 be the one selected as the furthest away. Then the second BFS will find that all vertices are 1 away from vertex 3, so will incorrectly conclude that the diameter of the graph is 1, when in actual fact it is 2.

\section*{Betweenness centrality and Newman-Girvan method examples}

\begin{enumerate}
    \item \begin{enumerate}
        \item The betweenness centralities are as follows:\\
            \begin{tabular}{|c|c|}
                \hline
                \textbf{Node} & \textbf{Betweenness centrality}\\
                \hline
                1 & 1.0 \\
                2 & 1.0 \\
                3 & 1.0 \\
                4 & 1.0 \\
                5 & 1.0 \\
                \hline
            \end{tabular}
        \item All 5 of the edges in this graph have the same betweenness centrality (3.0).
        \item You get a completely disconnected graph, i.e. the 5 vertices on their own without any edges between them.
    \end{enumerate}
    \item \begin{enumerate}
        \item The betweenness centralities are as follows:\\
            \begin{tabular}{|c|c|}
                \hline
                \textbf{Node} & \textbf{Betweenness centrality}\\
                \hline
                1 & 6.0 \\
                2 & 0.0 \\
                3 & 0.0 \\
                4 & 0.0 \\
                5 & 0.0 \\
                \hline
            \end{tabular}
        \item All 4 of the edges in this graph have the same betweenness centrality (4.0).
        \item You get a completely disconnected graph, i.e. the 5 vertices on their own without any edges between them.
    \end{enumerate}
    \item \begin{enumerate}
        \item The betweenness centralities are as follows:\\
            \begin{tabular}{|c|c|}
                \hline
                \textbf{Node} & \textbf{Betweenness centrality}\\
                \hline
                1 & 5.0 \\
                2 & 0.0 \\
                3 & 0.0 \\
                4 & 0.0 \\
                5 & 0.0 \\
                \hline
            \end{tabular}
        \item The (1, 4) and (1, 5) edges have highest betweenness centrality (5.0).
        \item We end up with one cluster and two lone vertices: [1, 2, 3] connected by the edges (1, 2), (2, 3), and (1, 3), and two lone vertices (4 and 5) disconnected from the rest of the graph, i.e. with no edges to/from either of them.
    \end{enumerate}
    \item \begin{enumerate}
        \item The betweenness centralities are as follows:\\
            \begin{tabular}{|c|c|}
                \hline
                \textbf{Node} & \textbf{Betweenness centrality}\\
                \hline
                1 & 0.0 \\
                2 & 3.0 \\
                3 & 0.0 \\
                4 & 3.0 \\
                5 & 0.0 \\
                \hline
            \end{tabular}
        \item The (1, 2), (2, 4) and (4, 5) edges have highest betweenness centrality (4.0).
        \item We end up with one cluster and two lone vertices: [2, 3, 4] connected by the edges (2, 3) and (3, 4), and two lone vertices (1 and 5) disconnected from the rest of the graph, i.e. with no edges to/from either of them.
    \end{enumerate}
    \item \begin{enumerate}
        \item The betweenness centralities are as follows:\\
            \begin{tabular}{|c|c|}
                \hline
                \textbf{Node} & \textbf{Betweenness centrality}\\
                \hline
                1 & 0.0 \\
                2 & 3.0 \\
                3 & 4.0 \\
                4 & 0.0 \\
                5 & 0.0 \\
                \hline
            \end{tabular}
        \item The (2, 3) edge has highest betweenness centrality (6.0).
        \item We end up with two clusters: [1, 2] (connected by the (1, 2) edge), and [3, 4, 5] (connected by the (3, 4), (4, 5), and (3, 5) edges).
    \end{enumerate}
    \item \begin{enumerate}
        \item The betweenness centralities are as follows:\\
            \begin{tabular}{|c|c|}
                \hline
                \textbf{Node} & \textbf{Betweenness centrality}\\
                \hline
                1 & 0.0 \\
                2 & 5.0 \\
                3 & 0.0 \\
                4 & 0.0 \\
                5 & 0.0 \\
                \hline
            \end{tabular}
        \item The (1, 2) and (2, 5) edges have highest betweenness centrality (4.0).
        \item We end up with one cluster and two lone vertices: [2, 3, 4] connected by the edges (2, 3), (3, 4), and (2, 4), and two lone vertices (1 and 5) disconnected from the rest of the graph, i.e. with no edges to/from either of them.
    \end{enumerate}
    \item \begin{enumerate}
        \item The betweenness centralities are as follows:\\
            \begin{tabular}{|c|c|}
                \hline
                \textbf{Node} & \textbf{Betweenness centrality}\\
                \hline
                1 & 3.0 \\
                2 & 1.0 \\
                3 & 0.0 \\
                4 & 1.0 \\
                5 & 0.0 \\
                \hline
            \end{tabular}
        \item The (1, 5) edge has highest betweenness centrality (4.0).
        \item We end up with one cluster and one lone vertex: [1, 2, 3, 4] connected by the edges (1, 2), (2, 3), (2, 4), (3, 4), and (1, 4), and one lone vertex (5) disconnected from the rest of the graph, i.e. with no edges to/from it.
    \end{enumerate}
\end{enumerate}

\section*{Random graphs and metrics}
TODO

\end{document}
